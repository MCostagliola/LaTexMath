% Hey There! This is a template for an engineering/math homework assignment.
% I created it because not many templates in the Overleaf library are structured this way.
% Feel free to use it or edit it as you see fit.

% Document Type:
\documentclass[12pt]{article}

% Packages Used (add more if needed):
\usepackage[utf8]{inputenc}
\usepackage{geometry}
\geometry{a4paper, margin=1in}
\usepackage{setspace}
\setstretch{1.5}
\usepackage{hyperref}
\hypersetup{
    colorlinks=true,
    linkcolor=blue,
    filecolor=blue,
    urlcolor=blue,
    citecolor=blue,
    pdfborder={0 0 0}
}
\usepackage{xcolor}
\usepackage{tcolorbox}
\usepackage{amsmath}
\usepackage{amssymb}
\usepackage{amsfonts}
\usepackage{graphicx} % For including graphics
\usepackage{fancyhdr}
\usepackage{tikz} % For Hasse diagram
\usepackage{pgfplots}
\pagestyle{fancy}
\fancyhf{}
\fancyfoot[C]{\thepage}

% Define Homework Number and Title (we will use them later).
\newcommand{\hwNumber}{\# 2}
\newcommand{\hwTitle}{Practice Exam}
\newcommand{\univName}{George Mason University}
\newcommand{\courseNum}{Fall 2024 --- Section 006}
\newcommand{\course}{Math 125: Discrete Math}

% Document Content: 
\begin{document}

%% Cover Page (replace everything in [] with your details): 
%%%%%%%%%%%%%%%%%%%%%%%%%%%%%%%%%%%%%%%%%%%%%%%%%%%%%%%%%%
%%%%%%%%%%%%%%%%%%%%START HERE%%%%%%%%%%%%%%%%%%%%%%%%%%%%
%%%%%%%%%%%%%%%%%%%%%%%%%%%%%%%%%%%%%%%%%%%%%%%%%%%%%%%%%%

\begin{center}
	\vspace*{2cm}
	\textbf{George Mason University \\}
	\textbf{Department of Mathematical Sciences \\}
	\vfill
	\textbf{\course\\}
	\textbf{Dr. Morris \\}
	\textbf{\courseNum \\}
	% Example for Semester [20YY][YY]: Spring 2023-24
	\textbf{\today \\}
	\textbf{\hwTitle\ \space \hwNumber\\}
	% Example: Homework 1
	\vfill
	\textbf{Matteo Costagliola \\}
	(Student ID: G01488318) \\
	\vfill
\end{center}

%% Numbering Starting Page 2
\thispagestyle{empty}
\newpage
\pagenumbering{arabic}

% Set Up Header
\fancyhead[L]{\textbf{\univName}} %University Name Should be Abbreviated. 
\fancyhead[C]{\textbf{\hwTitle\ \hwNumber}}
\fancyhead[R]{\textbf{\course}}

%% Start Writing Your Homework.

\section*{Problem 1}

\begin{tcolorbox}[colback=gray!10, colframe=black, title=Given]
	% Insert the concise given of your problem here
	Suppose that \( A = \{1,2,3\} \text{ and } B = \{1,3,5,7\} \). Find:
	\begin{enumerate}
		\item [(a)] \( A \times B \)
		\item [(b)] \( \mathcal{P}(A) \text{, the power set of } A \)
	\end{enumerate}
\end{tcolorbox}

\begin{tcolorbox}[colback=yellow!20, colframe=black, title=Solution]
	% Insert the solution of your problem here
	\begin{enumerate}
		\item [(a)] \( A \times B = \{(1,1), (1,3), (1,5), (1,7), (2,1), (2,3), (2,5), (2,7), (3,1), (3,3), (3,5), (3,7)\} \)
		\item [(b)] \( \mathcal{P}(A) = \{\emptyset, \{1\}, \{2\}, \{3\}, \{1,2\}, \{1,3\}, \{2,3\}, \{1,2,3\} \} \)
	\end{enumerate}
\end{tcolorbox}
%%%%%%%%%%%%%%%%%%%%%%%%%%%%%%%%%%%%%%%%%%%%%%%%%%%%%%%%%%%%%%%%
\section*{Problem 2}

\begin{tcolorbox}[colback=gray!10, colframe=black, title=Given]
	% Insert the concise given of your problem here
	Prove or give a counterexample to the statement: For all sets \( A, B, C, (A - B) \cup C = A - (B \cap C) \).
\end{tcolorbox}

\begin{tcolorbox}[colback=yellow!20, colframe=black, title=Solution]
	% Insert the solution of your problem here
	We propose a counterexample:
	Let \( A = \{1\}, B = \{2\}, \text{ and } C = \emptyset \)
	\\
	\( (A - B) \cup C = \)
	\[ A - B = 1 \]
	\[ (A - B) \cup C = \{\emptyset, {1} \} \]
	\( A - (B \cap C) = \)
	\[ B \cap C = \emptyset \]
	\[ A - (B \cap C) = 1 \]
	\\Thus we have \( \{\emptyset, \{1\}\} \neq 1 \).
\end{tcolorbox}
%%%%%%%%%%%%%%%%%%%%%%%%%%%%%%%%%%%%%%%%%%%%%%%%%%%%%%%%%%%%%%%%%%
\section*{Problem 3}

\begin{tcolorbox}[colback=gray!10, colframe=black, title=Given]
	% Insert the concise given of your problem here
	Let \( f=\{(a,a), (b,c), (c,b)\} \text{ and } g=\{(a,c), (b,c), (c,a)\}\) be two functions from \( \{a,b,c\} \text{ to } \{a,b,c\} \).
	\begin{enumerate}
		\item [(a)] Find \( f^{-1} \). Does $g$ have an inverse?
		\item [(b)] Find \( g \circ f \).
	\end{enumerate}
\end{tcolorbox}

\begin{tcolorbox}[colback=yellow!20, colframe=black, title=Solution]
	% Insert the solution of your problem here
	\begin{enumerate}
		\item [(a.1)] \( f^{-1} = \{(a,a), (c,b), (b,c)\}\)
		\item [(a.2)] No $g$ does not have an inverse as $g$ is not a bijective function.
		\item [(b)] \( g \circ f = \{(a,c), (b,a), (c,c) \} \)
	\end{enumerate}
\end{tcolorbox}
%%%%%%%%%%%%%%%%%%%%%%%%%%%%%%%%%%%%%%%%%%%%%%%%%%%%%%%%%%%%%%%%%%%%%%%%%
\section*{Problem 4}

\begin{tcolorbox}[colback=gray!10, colframe=black, title=Given]
	% Insert the concise given of your problem here
	Define \( f : \{x\in \mathbb{R} \mid x>0\} \mapsto \{y\in \mathbb{R} \mid y>1\}\text{ by }f(x)=1+\frac{1}{x}\).
	\begin{enumerate}
		\item [(a)] Show that $f$ is one-to-one and onto.
		\item [(b)] Find the inverse function $f^{-1}(x)$.
	\end{enumerate}
\end{tcolorbox}

\begin{tcolorbox}[colback=yellow!20, colframe=black, title=Solution]
	% Insert the solution of your problem here
	\begin{enumerate}
		\item [(a.1)] \textbf{Proof: $f$ is one-to-one.}
		      \\Let $x_1$, $x_2$ $\in$ $\mathbb{R}$ such that \( f(x_1)=f(x_2) \), then by the definition of $f$
		      \[ 1 + \frac{1}{x_1} = 1 + \frac{1}{x_2} \]
		      Then, by simple algebra
		      \[ \frac{1}{x_1}=\frac{1}{x_2}, \; x_1=x_2 \]
		      Thus we have that $f$ is one-to-one.
		\item [(a.2)] \textbf{Proof: $f$ is onto.}
		      \\Let $y$ be an element of the range of $f$ such that there exists an $x$ where $f(x)=y$.
		      \\Then by the rule of $f$
		      \[y=1+\frac{1}{x},\; y-1=\frac{1}{x},\; x=\frac{1}{y-1} \]
		      Plugging x into $f(x)=y$ gives
		      \[f(\frac{1}{y-1})=1+\frac{1}{\frac{1}{y-1}}=1+y-1=y=y \]
		      Thus $f$ is onto.
		\item [(b)] $f^{-1}=$
		      \[y=1+\frac{1}{x},\; x=1+\frac{1}{y},\; x-1=\frac{1}{y},\; y=\frac{1}{x-1}\]
		      \[f^{-1}=\frac{1}{x-1}\]
	\end{enumerate}
\end{tcolorbox}
%%%%%%%%%%%%%%%%%%%%%%%%%%%%%%%%%%%%%%%%%%%%%%%%%%%%%%%%%%%%%%%%%%%%%%%%%
\section*{Problem 5}

\begin{tcolorbox}[colback=gray!10, colframe=black, title=Given]
	% Insert the concise given of your problem here
	Define a transitive relation $R$ on $\mathbb{Z}$ by $xRy$ if and only if 3 divides $x-y$.
	Prove that $R$ is transitive.
\end{tcolorbox}

\begin{tcolorbox}[colback=yellow!20, colframe=black, title=Solution]
	% Insert the solution of your problem here
	\textbf{Proof: $R$ is transitive.}
	\\Let $x,y,x\in R \text{ such that } xRy\text{ and }yRz$. Then by the definition of $R$
	we have
	\[x-y=3m \text{ for some integer $m$}\] \[y-z=3l \text{ for some integer $l$} \]
	For $R$ to be transitive we must have $xRz$, so we solve the second equation for $y$
	and substitute this into the first equation.
	\[y=3l+z\] \[x-(3l+z)=3m,\; x-3l-z=3m\] \[x-z=3m+3l,\; x-z=3(m+l), \text{ where $(m+l)\in \mathbb{Z}$}\]
	Thus $R$ is transitive.
\end{tcolorbox}
%%%%%%%%%%%%%%%%%%%%%%%%%%%%%%%%%%%%%%%%%%%%%%%%%%%%%%%%%%%%%%%%%%%%%%%%%
\section*{Problem 6}

\begin{tcolorbox}[colback=gray!10, colframe=black, title=Given]
	% Insert the concise given of your problem here
	Let $R=\{(1,2),(2,1),(2,2),(1,1),(3,3)\}$ be a relation on $\{1,2,3,4\}$.
	For each of the properties: reflexivity, symmetry, transitivity, antisymmetry, state whether or not $R$ has the property.
\end{tcolorbox}

\begin{tcolorbox}[colback=yellow!20, colframe=black, title=Solution]
	% Insert the solution of your problem here
	$R$ is not reflexive as $(4,4)\notin R$.\\
	$R$ is symmetric as $(a,b)\in R \text{ if } (b,a) \text{ is in } R$.\\
	$R$ is transitive as $(a,b)\in R \land (b,c)\in R \implies (a,c)\in R$.\\
	$R$ is not antisymmetric as $(1,2)\in R \text{ and } (2,1)\in R \text{ but } 1\neq2$.
\end{tcolorbox}
%%%%%%%%%%%%%%%%%%%%%%%%%%%%%%%%%%%%%%%%%%%%%%%%%%%%%%%%%%%%%%%%%%%%%%%%%
\section*{Problem 7}

\begin{tcolorbox}[colback=gray!10, colframe=black, title=Given]
	% Insert the concise given of your problem here
	Let $A=\{a,b,c,d\} \text{ and let } R=\{(a,a),(a,c),(b,b),(c,c),(c,a),(d,d)\}$
	\begin{enumerate}
		\item [(a)] Prove that $R$ is an equivalence relation on $A$.
		\item [(b)] What are the equivalence classes for $R$?
	\end{enumerate}
\end{tcolorbox}

\begin{tcolorbox}[colback=yellow!20, colframe=black, title=Solution]
	% Insert the solution of your problem here
	\begin{enumerate}
		\item [(a)] $R$ is reflexive as $\forall x\in A, (x,x)\in R$. $R$ is symmetric as $\forall (x,y)\in R, (y,x)\in R$.
		      $R$ is transitive as $\forall x,y,x\in R, (x,y)\in R \land (y,z)\in R \implies (x,z)\in R$.
		\item [(b)] The equivalence classes for $R$ are as follows
		      \[\{a,c\},\; \{b\},\; \{d\}\]
	\end{enumerate}
\end{tcolorbox}
%%%%%%%%%%%%%%%%%%%%%%%%%%%%%%%%%%%%%%%%%%%%%%%%%%%%%%%%%%%%%%%%%%%%%%%%%
\section*{Problem 8}

\begin{tcolorbox}[colback=gray!10, colframe=black, title=Given]
	% Insert the concise given of your problem here
	Consider the following partial order on\\
	$\{a,b,c,d\}:\;{(a,a),(b,b),(c,c),(a,b),(a,c),(d,c),(d,d)}$.
	\begin{enumerate}
		\item [(a)] Draw the Hasse diagram of the partial order.
		\item [(b)] List the minimal and the least elements.
	\end{enumerate}
\end{tcolorbox}

\begin{tcolorbox}[colback=yellow!20, colframe=black, title=Solution]
	% Insert the solution of your problem here
	\begin{enumerate}
		\item [(a)] \begin{tikzpicture}
			      \node (a) at (2,0) {$a$};
			      \node (b) at (3,1) {$b$};
			      \node (c) at (1,1) {$c$};
			      \node (d) at (0,0) {$d$};

			      \draw [black,  thin, shorten <=-2pt, shorten >=-2pt] (d) -- (c) -- (a) -- (b);
		      \end{tikzpicture}
		\item [(b)] The minimal elements are: $a$ and $c$, as no elements are below these. There is no least element in this partial order.
	\end{enumerate}
\end{tcolorbox}
%%%%%%%%%%%%%%%%%%%%%%%%%%%%%%%%%%%%%%%%%%%%%%%%%%%%%%%%%%%%%%%%%%%%%%%%%
\section*{Problem 9}

\begin{tcolorbox}[colback=gray!10, colframe=black, title=Given]
	% Insert the concise given of your problem here
	\begin{enumerate}
		\item [(a)] How many odd integers are there from 3000 through 9999?
		\item [(b)] How many odd integers from 3000 through 9999 have distinct digits?
	\end{enumerate}
\end{tcolorbox}

\begin{tcolorbox}[colback=yellow!20, colframe=black, title=Solution]
	% Insert the solution of your problem here
	\begin{enumerate}
		\item [(a)] $N(9999-3000)=7000,\; \frac{7000}{2}=3500$, thus there are 3500 odd integers from 3000 to 9999.
		\item [(b)] We will divide this problem into two cases. First, consider if the last digit is 1.
		      This leaves us with 7 choices for the first digit, 8 choices for the second, and 7 choices for the third.
		      For a total of 392 integers. For the second case, consider numbers ending with 3,5,7, and 9.
		      There are 4 choices for the last digit, 6 choices for the first, 8 choices for the second, and 7 choices for the third.
		      For a total of 1334 integers. Giving a total between the two cases of 1736 odd integers with distinct digits
		      between 3000 and 9999.
	\end{enumerate}
\end{tcolorbox}
%%%%%%%%%%%%%%%%%%%%%%%%%%%%%%%%%%%%%%%%%%%%%%%%%%%%%%%%%%%%%%%%%%%%%%%%%
\section*{Problem 10}

\begin{tcolorbox}[colback=gray!10, colframe=black, title=Given]
	% Insert the concise given of your problem here
	Suppose that $A,B,\text{ and }C$ are sets and that $N(A)=28,\; N(B)=26,\; N(C)=14,\;N(A\cap B)=8,\; N(A\cap C)=4
		N(B\cap C)=3,\text{ and }N(A\cap B\cap C)=2$. What is $N(A\cup B\cup C)$?
\end{tcolorbox}

\begin{tcolorbox}[colback=yellow!20, colframe=black, title=Solution]
	% Insert the solution of your problem here
	$N(A\cup B\cup C)=28+26+14-8-4-3+2=55.$
\end{tcolorbox}
%%%%%%%%%%%%%%%%%%%%%%%%%%%%%%%%%%%%%%%%%%%%%%%%%%%%%%%%%%%%%%%%%%%%%%%%%


% Indicate that your document has ended.

\vfill
\begin{center}
	\rule{0.5\linewidth}{0.5pt}\\
	\textit{End of \textbf{\hwTitle \space \hwNumber}}
\end{center}

\end{document}

% Hope this template was helpful!
% Good luck :)