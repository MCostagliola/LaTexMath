\documentclass[12pt]{article}

\usepackage{changepage, amsmath, amssymb, pgfplots, tikz}
\pgfplotsset{compat=1.18}
\usepackage{fullpage}
\usepackage{darkmode}
\enabledarkmode
\usepackage{ulem}

%%%%%%%%%%%%%%%%%%%%%%%%%%%%%%%%%%%%%%%%%%%%%%%%%%%%%%%%%%%%%
% Custom title section
\newcommand{\classinfo}[3]{
    \begin{center}
        {\Large \textbf{#1}} \\ % Class name
        \vspace{0.5cm}
        {\Large #2} \\ % Lecture topic
        \vspace{0.5cm}
        {\large #3}   % Date
    \end{center}
    \vspace{0.5cm} % Add spacing below the title
}
%%%%%%%%%%%%%%%%%%%%%%%%%%%%%%%%%%%%%%%%%%%%%%%%%%%%%%%%%%%%%
% Define custom theorem command with optional numbering and inline separator
\newcommand{\thm}[3][]{ % [#1] is optional, #2 is title, #3 is content
    \noindent\textbf{#2} % Title in bold
    \ifx&#1& % Check if optional input is empty
        % No optional input, skip number
    \else
        \ \textbf{#1} % Include the optional number, bold with space
    \fi
    \ --- % Add the separator
    #3 % Theorem content
}
%%%%%%%%%%%%%%%%%%%%%%%%%%%%%%%%%%%%%%%%%%%%%%%%%%%%%%%%%%%%%
% Define custom claim command with optional numbering and inline separator
\newcommand{\clm}[3][]{ % [#1] is optional, #2 is title, #3 is content
    \noindent\textbf{#2} % Title in bold
    \ifx&#1& % Check if optional input is empty
        % No optional input, skip number
    \else
        \ \textbf{#1} % Include the optional number, bold with space
    \fi
    \ --- % Add the separator
    #3 % claim content
}
%%%%%%%%%%%%%%%%%%%%%%%%%%%%%%%%%%%%%%%%%%%%%%%%%%%%%%%%%%%%%
% Define custom corollary command with optional numbering and inline separator
\newcommand{\crl}[3][]{ % [#1] is optional, #2 is title, #3 is content
    \noindent\textbf{#2} % Title in bold
    \ifx&#1& % Check if optional input is empty
        % No optional input, skip number
    \else
        \ \textbf{#1} % Include the optional number, bold with space
    \fi
    \ --- % Add the separator
    #3 % corollary content
}
%%%%%%%%%%%%%%%%%%%%%%%%%%%%%%%%%%%%%%%%%%%%%%%%%%%%%%%%%%%%%
% Define custom lemma command with optional numbering and inline separator
\newcommand{\lma}[3][]{ % [#1] is optional, #2 is title, #3 is content
    \noindent\textbf{#2} % Title in bold
    \ifx&#1& % Check if optional input is empty
        % No optional input, skip number
    \else
        \ \textbf{#1} % Include the optional number, bold with space
    \fi
    \ --- % Add the separator
    #3 % lemma content
}
%%%%%%%%%%%%%%%%%%%%%%%%%%%%%%%%%%%%%%%%%%%%%%%%%%%%%%%%%%%%%
% Custom proof command
\newcommand{\proof}[2]{ % #1 is the proof content
    \vspace{0.3cm} % Add space before the proof
    \noindent\textit{#1} % "Proof" in italics
    \hspace{0.2cm} #2 % Proof content
    \hfill$\square$ % QED symbol at the end
    \vspace{0.3cm} % Add space after the proof
}
%%%%%%%%%%%%%%%%%%%%%%%%%%%%%%%%%%%%%%%%%%%%%%%%%%%%%%%%%%%%%

%%%%%%%%%%%%%%%%%%%%%%%%%%%%%%%%%%%%%%%%%%%%%%%%%%%%%%%%%%%%%
\begin{document}
\classinfo{Math 125: Discrete Math I}{Introduction to Graph Theory}{November 25, 2024}

\section*{\underbar{Binomial Theorem}}
\thm{The Binomial Theorem}{Given any real numbers $a$ and $b$ and any nonnegative integer n,\\\\
${(a+b)}^{n}=\sum_{k=0}^{n}\binom{n}{k}a^{n-k}b^{k}$.}
\\\\
For example, ${(a+b)}^{3}\\\\=\binom{3}{0}a^{3}b^{0}+\binom{3}{1}a^{2}b^{1}+\binom{3}{2}a^{1}b^{2}+\binom{3}{3}a^{0}b^{3}\\\\=a^{3}+3a^{2}b+3ab^{2}+b^{3}$.\\\\
$(a+b)^{3}=(a+b)(a+b)(a+b)$ multiplied out is the sum of all possible ordered products of $a$'s and $b$'s. It's $aaa+baa+aba+aab+bba+bab+abb+bbb$. The variable in the $i$th position is taken from the $i$th parenthesis.\\\\
\proof{Inductive Proof of Binomial Theorem}{
Clearly, ${(a+b)}^{0}=1=\sum_{k=0}^{0}\binom{0}{k}a^{0-k}b^{0}$.\\\\
Suppose ${(a+b)}^{m}=\sum_{k=0}^{m}\binom{m}{k}a^{m-k}b^{k}$ for some integer $m\geq 0$.\\\\
Then ${(a+b)}^{m+1}={(a+b)}{(a+b)}^{m}\\\\=(a+b)\sum_{k=0}^{m}\binom{m}{k}a^{m-k}b^{k}\\\\=\sum_{k=0}^{m}\binom{m}{k}a^{m-k+1}b^{k}+\sum_{j=0}^{m}\binom{m}{j}a^{m-j}b^{j+1}$.\\\\
For the second sum, let $k=j+1$ so that the sum becomes\\\\
$=\sum_{k=0}^{m}\binom{m}{k}a^{m-k+1}b^{k}+\sum_{k=1}^{m+1}\binom{m}{k-1}a^{m-k+1}b^{k}\\\\=\binom{m+1}{0}a^{m+1}b^{0}+\sum_{i=1}^{m}[\binom{m}{k}+\binom{m}{k-1}]a^{m+1-k}b^{k}+\binom{m+1}{m+1}a^{0}b^{m+1}\\\\=\sum_{k=0}^{m+1}\binom{m+1}{k}a^{m+1-k}b^{k}$. By the principle of mathematical induction, the binomial theorem is true for all $n\geq 0$.
}

\end{document}